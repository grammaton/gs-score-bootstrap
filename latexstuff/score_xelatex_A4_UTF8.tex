%!TEX TS-program = xelatex
%!TEX encoding = UTF-8 Unicode

\documentclass[10pt, a4paper, twoside]{book}
\usepackage[margin=1in]{geometry}

\usepackage[parfill]{parskip}    % Activate to begin paragraphs with an empty line rather than an indent
\usepackage{graphicx}
\usepackage{amssymb}
\usepackage{wrapfig}

\usepackage{paralist}
\usepackage{array}
\usepackage{longtable}
\usepackage{multirow}

\usepackage{url}

\usepackage{multicol}
\usepackage{pdfpages}

%-------------------------------------------------------------
%----------------------------------------------------- FONTS -%-------------------------------------------------------------

\linespread{1.04}

\usepackage{fontspec,xltxtra,xunicode}
\defaultfontfeatures{Mapping=tex-text}
\setromanfont[Mapping=tex-text]{IM FELL Great Primer}
\setsansfont[Scale=MatchLowercase,Mapping=tex-text]{Gill Sans}
\setmonofont[]{Fira Mono}

\newfontfamily{\scaps}{IM FELL Great Primer SC}
\newfontfamily{\grafs}{IM FELL FLOWERS 1}
\usepackage[italian, english]{babel}

\usepackage{lilyglyphs}

\usepackage[hang, small,labelfont=bf,up,textfont=it,up]{caption}

%-------------------------------------------------------------
%---------------------------------------------------- HEADER -%-------------------------------------------------------------

\usepackage{fancyhdr}

\fancypagestyle{plain}{%
\fancyhf{} % clear all header and footer fields
%\fancyhead[CO,CE]{---Draft---}
\fancyfoot[RO,LE]{\fontsize{18pt}{18pt}\selectfont\thepage} % except the center
\renewcommand{\headrulewidth}{0pt}
\renewcommand{\footrulewidth}{0pt}}
\pagestyle{plain}

%-------------------------------------------------------------
%----------------------------------------------------- TITLE -%-------------------------------------------------------------

% customs |ˈkʌstəmz|
% pluralnoun
% the official department that administers and collects the duties levied by a government on imported goods: cocaine seizures by customs have risen this year | [ as modifier ] : a customs officer.
% • the place at a port, airport, or frontier where officials check incoming goods, travellers, or luggage: we were through customs with a minimum of formalities.

\newcommand{\theName}{Georg Friedrich HAAS} % autore
\newcommand{\theTitle}{Bariton-Saxophon-Konzert} % titolo
\date{2008} % nessuna data
\newcommand{\scoredFor}{for baritone saxophone and orchestra} % scored for
\newcommand{\instrDetails}{ % instrumentation details, separate each by \newline
  1st flute \newline
  2nd flute \newline
  3rd flute (+picc) \newline
  1st oboe \newline
  2nd oboe \newline
  3rd oboe (+c.a) \newline
  1st clarinet in Bb \newline
  2nd clarinet in Bb \newline
  3rd clarinet in A \newline
  1st bassoon \newline
  2nd bassoon \newline
  3rd bassoon (+cbsn) \newline
  1st horn in F \newline
  2nd horn in F \newline
  3rd horn in F \newline
  4th horn in F \newline
  1st trumpet in Bb \newline
  2nd trumpet in Bb \newline
  3rd trumpet in Bb \newline
  1st trombone \newline
  2nd trombone \newline
  3rd trombone \newline
  bass trombone \newline
  bass tuba \newline
  1st percussion \newline
  2nd percussion \newline
  3rd percussion \newline
  violin I (1st, 2nd player) \newline
  violin I (3rd, 4th player) \newline
  violin I (5th, 6th player) \newline
  violin I (7th, 8th player) \newline
  violin I (9th, 10th player) \newline
  violin I (11th, 12th player) \newline
  violin I (13th, 14th player) \newline
  violin II (1st, 2nd player) \newline
  violin II (3rd, 4th player) \newline
  violin II (5th, 6th player) \newline
  violin II (7th, 8th player) \newline
  violin II (9th, 10th player) \newline
  violin II (11th, 12th player) \newline
  viola (1st, 2nd player) \newline
  viola (3rd, 4th player) \newline
  viola (5th, 6th player) \newline
  viola (7th, 8th player) \newline
  viola (9th, 10th player) \newline
  violoncello (1st, 2nd player) \newline
  violoncello (3rd, 4th player) \newline
  violoncello (5th, 6th player) \newline
  violoncello (7th, 8th player) \newline
  contrabass (1st, 2nd player) \newline
  contrabass (3rd, 4th player) \newline
  contrabass (5th, 6th player) \newline
  }
\newcommand{\commBy}{Kompositionsauftrag des Westdeutschen Rundfunks} % commissioned by
\newcommand{\durAtion}{25 minutes} % duration
\newcommand{\dedicAtion}{für Akira und Michiko Ueda} % dedication
 % file con le impostazioni personali

\title{{
	\fontsize{19}{10}\selectfont{
	\theName
	} \\
	\bigskip {
	\fontsize{81}{10}\selectfont{
	\vspace{6cm}\theTitle
	}}}}

\author{
	\fontsize{19}{10}\selectfont{
	\textit{
		\scoredFor
	}}}

\date{
	\emph{{
	\LARGE \date
	}}\vfill
	%GS006-E002
	}
	
	

\makeatletter
%\frameattext{<backgroundcolor>}{linecolor}{<linewidth>}
\newdimen\extraxsep
\newdimen\extraysep
\extraxsep=20mm
\extraysep=20mm
\newcommand\frameattext[3]{%
  \linethickness{#3}%
  \AddToShipoutPicture*{%
    \AtTextLowerLeft{%text-boder
       \put(\LenToUnit{-,5\extraxsep},\LenToUnit{-0.5\extraysep}){\color{#1}%
             \rule{\dimexpr\textwidth+\extraxsep\relax}{\dimexpr\textheight+\extraysep\relax}}%
       \put(\LenToUnit{-,5\extraxsep},\LenToUnit{-0.5\extraysep}){\color{#2}%
       \framebox(\LenToUnit{\dimexpr\textwidth+\extraxsep\relax},%
                 \LenToUnit{\dimexpr\textheight+\extraysep\relax}){}
       }
    }%
  }%
}
%\frameatpage{<backgroundcolor>}{linecolor}{<linewidth>}
\newcommand\frameatpage[3]{%
  \linethickness{#3}%
  \AddToShipoutPicture*{%
    \AtPageLowerLeft{%page-border
      \put(0,0){\color{#1}\rule{\paperwidth}{\paperheight}}%
      \put(\LenToUnit{\@wholewidth},\LenToUnit{\@wholewidth}){%
       \color{#2}\framebox(\LenToUnit{\dimexpr\paperwidth-2\@wholewidth\relax},%
                  \LenToUnit{\dimexpr\paperheight-2\@wholewidth\relax}){}%
      }%
    }%
  }%
}

\makeatother

%-------------------------------------------------------------
%-------------------------------------------------- DOCUMENT -%-------------------------------------------------------------

\begin{document}

%\AddToShipoutPictureBG*{%
%	\put(-5,-5){
%	\includegraphics[keepaspectratio]{onde}}
%  }

\setlength{\columnsep}{.4in}



\frameattext{white}{black}{2pt}

\begin{center}
	~\\
	\vfill
    \fontsize{19}{10}\selectfont{Giuseppe SILVI} \\
		\vspace{2cm}
    \fontsize{81}{10}\selectfont{A. SAX.} \\
		\vspace{.5cm}
	\fontsize{19}{10}\selectfont{\emph{for any kind of Saxophone \& Electronics}} \\
		\vspace{.5cm}
	\fontsize{13}{10}\selectfont{2013--2015} \\
		\vspace{16cm}
	\fontsize{13}{10}\selectfont{GS006-E002} \\
\end{center}
%\maketitle

\thispagestyle{empty}



\clearpage

\thispagestyle{empty}

~\vfill

\includegraphics[width=.25\columnwidth]{images/by-nc-sa}\\
\emph{A. Sax.} by Giuseppe Silvi is licensed under a Creative Commons \\
Attribution-NonCommercial-ShareAlike 4.0 International License.\\ 
Permissions beyond the scope of this license may be available at\\
giuseppesilvi.com/asax.%\marginpar{prova}


\clearpage

% ---------------------------------------------
% ------------------------------- ITALIANO ----
% ---------------------------------------------

\section*{Istruzioni per l'esecuzione}

\begin{multicols}{2}

Il brano è composto di otto moduli nominati dalla lettera \emph{A} alla lettera \emph{H}. L'esecutore ne sceglie quattro e li dispone nell'ordine che preferisce tra l'\emph{introduzione} ed il \emph{finale} dati. È lasciata alla discrezione dell'esecutore la scelta del tipo di Sax o dei diversi Sax da utilizzare durante il brano o parte di esso.

Ognuna delle otto pagine musicali è dedicata ad un brano (e di conseguenza ad un autore) diverso del repertorio per sassofono solo. L'\emph{introduzione} ed il \emph{finale} fanno da contenitori entro cui questi moduli vengono montati. Ognuno degli otto moduli esprime tratti stilistici propri che si sviluppano nel contesto sonoro elettroacustico nascondendo e svelando le proprie origini. Le scelte possibili, di quattro pannelli su otto, senza ripetizioni, sono 1680. 

Ognuno degli otto moduli dura novanta secondi, divisi in sei sistemi di circa quindici secondi ciascuno. L'\emph{introduzione} ed il \emph{finale} hanno entrambi durata di un minuto. La durata totale è di otto minuti.

La partitura è trascritta indicando le posizioni condivise dall'intera famiglia dei saxofoni così che la scelta del Sax ne condizioni i suoni d'effetto. \\
Le alterazioni si riferiscono solo alla nota che precedono, il~ {\large \natural~} ha la sola funzione di facilitare la lettura.

Suonare sempre senza vibrare.

L'elettronica consiste in una parte fissa\footnote{L'elettronica, sia nella traccia fissa che negli algoritmi per il \emph{live electronics} sono disponibili all'indirizzo web: \newline \url{http://giuseppesilvi.com/asax}}, della durata di otto minuti, ovvero l'intera durata del brano, ed una parte live in relazione ai moduli scelti dal saxofonista. Il brano inizia e termina con il solo suono del Sax, ciò è garantito da silenzi in testa ed in coda all'elettronica fissa.

L'elettronica è concepita attorno alle teorie di \emph{Forma Sonora} e \emph{Spazio Elettroacustico Omnidirezionale} che caratterizzano la ricerca elettroacustica e la produzione musicale dell'autore\footnote{Per \emph{Forma Sonora} si intende la forma percepita di un determinato oggetto sonoro quale può essere uno strumento acustico. Lo \emph{Spazio Elettroacustico Omnidirezionale} è realizzato mediante l'utilizzo di un sistema di diffusione tetraedrico, ovvero un altoparlante avente quattro facce triangolari ed in grado di diffondere suono in tutte le direzioni dello spazio. Maggiori informazioni sono disponibili all'indirizzo web: \newline \url{http://giuseppesilvi.com}}. Le due componenti elettroniche, quella fissa e quella dal vivo, utilizzano dati di analisi della forma sonora del Sax per creare un ambiente sonoro elettroacustico coerente. 

L'esecuzione elettronica del brano può essere affidata ad un esecutore elettroacustico o realizzata dallo stesso saxofonista mediante controlli a pedale.

\end{multicols}

\vfill

\begin{center}{
	\fontsize{35pt}{50pt}\selectfont{
	\bigskip {
	\grafs FYYH
	}}}
\end{center}

\vfill

\section*{Note di programma}

\begin{multicols}{2}



\emph{A. Sax.} è stato concepito per essere suonato da ogni tipo di Saxofono. Composto in occasione del \emph{Monaco Electroacoustique 2013}, festival internazionale di musica elettroacustica del Principato di Monaco, il brano è un omaggio sia al Saxofono che ad Adolphe Sax, suo genio creatore.

La prima esecuzione del brano è stata di Edoardo Capparucci, presso l'Auditorium dell'{\scaps Académie Rainier III} del Principato di Monaco, il 2 maggio 2013. Di Edoardo Capparucci  sono anche i suoni di sax elaborati nella perte elettronica fissa. 

Per \emph{Suona Italiano} il primo dicembre 2014 {\scaps A. Sax.} è stato eseguito da Enzo Filippetti presso la {\scaps Recital Hall} del {\scaps Conservatorio di Birmingham} e presso il {\scaps Performance Space} della {\scaps City University} di Londra il 9 dicembre. 

L'interpretazione di Enzo Filippetti e gli stimoli, da egli forniti nell'analisi delle prime stesure della partitura, hanno portato alla versione attuale qui pubblicata. Ad Enzo, incarnazione della passione per lo strumento, la ricerca e la didattica, va la mia più sincera gratitudine. 

%\flushright{------------------}
\vfill
~
\vfill

\end{multicols}

\clearpage

\section*{Regia informatica}

\begin{multicols}{2}

Lo scopo ultimo della regia informatica per il brano \emph{A. Sax.} è quello di descrivere l'ambiente sonoro come fosse un'evoluzione dello strumento acustico. L'elettronica evidenzia le caratteristiche timbrico-spaziali dello strumento senza mai prendere il sopravvento su quest'ultimo. Non uno sfondo ma, sfruttandone la \emph{forma sonora}, piuttosto, un alter-ego, uno sguardo altro che ne fissa le caratteristiche nel tempo e nello spazio.

Per ottenere ciò è necessario osservare che lo strumento dal vivo va amplificato solo se l'ambiente di esecuzione lo renda necessario e mai in misura per cui l'amplificazione superi il suono acustico dello strumento. 

%L'intera struttura elettronica è descritta in \emph{A-Format}\footnote{Secondo le teorie \emph{Ambisonic} di Michael Gerzon, lo spazio sonoro \emph{A-Format} è descritto da quattro componenti spaziali \emph{LF, RF, LB, RB}. Dal sistema \emph{A-Format} è possibile ricavare il \emph{B-Format} da cui è possibile generare diversi scenari possibili per la diffusione elettroacustica.} così da essere riproducibile senza decodifiche mediante il sistema di diffusione omni-direzionale \emph{S.T.ONE}\footnote{\url{http://giuseppesilvi.com/asax}}. 

Il brano può essere eseguito utilizzando molteplici tagli di sax, quindi la microfonazione può essere realizzata con due o più microfoni posizionati in modo da coprire l'intera gamma timbrica degli strumenti. I microfoni vengono quindi mixati in modo da ottenere un segnale monofonico da inviare ai processi di segnale.

Ogni processore ha controlli esecutivi del segnale in ingresso ma non in uscita, dove si hanno solo controlli operativi. In questo modo una volta tarate le uscite dei singoli processi si interviene solo sul segnale in entrata. Se tutti i controlli in entrata ai processi sono chiusi si ha la sola amplificazione dello strumento, che in ogni caso dovrà essere solo di rinforzo, se necessaria, in relazione all'ambiente esecutivo.

Il livello di uscita dei processi, considerato un suono al massimo del livello di entrata, non dovrà mai superare il livello percepito del suono acustico in sala. 

\end{multicols}

\subsection*{Algoritmi}

\begin{multicols}{2}

[d i s e g n a r e  a l g o r i t m i]
\end{multicols}

\clearpage

%\subsection*{algoritmo tikz}
%
%%\subsection*{Diffusione nello spazio}
%
%%\begin{center}
%%{\fontsize{20pt}{100pt}\selectfont{\bigskip  {\grafs o o o }}}
%%\end{center}
%
%
%
%\includegraphics[width=0.9\textwidth]{images/diagrammi.pdf}
%
%\cleardoublepage

%% ---------------------------------------------
%% ------------------------------- INGLESE -----
%% ---------------------------------------------
%
%\section*{Performance Instruction}
%
%\begin{multicols}{2}
%\end{multicols}
%
%\section*{Programme Note}
%
%\begin{multicols}{2}
%
%Il brano conta otto moduli nominati da A a H. L'esecutore ne sceglie quattro e li dispone nell'ordine che preferisce tra l'introduzione dil finale dati. È inoltre lasciato alla discrezione dell'esecutore la scelta del Sax o dei diversi Sax da utilizzare durante il brano o parte di esso. 
%
%La partitura non \'e scritta in suoni reali.
%
%Ognuna delle otto pagine é 
%
%\end{multicols}
%
%%\section*{Translation of the instructions from the score}
%
%%\end{multicols}

\clearpage

%!TEX TS-program = xelatex
%!TEX encoding = UTF-8 Unicode
% !TEX root = ./asax_score_xelatex_UTF8.tex

\section*{Legenda}

%+++++++++++++++++++++++++++++++++++++

\begin{table}[ht]
%\caption{default}
\begin{center}{\small
\begin{tabular}{m{6,1cm}m{9cm}}

\includegraphics[]{../lilypondstuff/LEGENDA/01intro.pdf}
&
{\scaps Suono di denti}.\newline  Appoggiando i denti sulla superficie dell'ancia si generano suoni molti acuti.
Trovare il suono più acuto possibile e tenerlo \emph{fortissimo}, senza vibrare né
modulare l'ampiezza. \\

\includegraphics[]{../lilypondstuff/LEGENDA/02finale.pdf}
&
{\scaps Trillo Grave}.\newline Effettuare il \emph{trillo} più grave possibile, \emph{diminuendo}, fino a
far scomparire il timbro dello strumento e lasciare il suono delle meccaniche.
Il \emph{trillo grave} del saxofono è l'ultimo suono del brano, tutta l'elettronica
deve scomparire prima che il trillo sia consumi. \\

\includegraphics[]{../lilypondstuff/LEGENDA/03Aglissato.pdf}
&
{\scaps Glissato irregolare}.\newline Si procede nella direzione del glissato
alternando intervalli corti a intervalli lunghi. \\

\includegraphics[]{../lilypondstuff/LEGENDA/04Cglissarmonico.pdf}
&
{\scaps Armonico portato}.\newline Tenendo fissa la posizione, \emph{portare} in maniera graduale il suono \emph{armonico} al \emph{fondamentale} e viceversa. \\

\includegraphics[]{../lilypondstuff/LEGENDA/05Cvoce.pdf}
&
{\scaps Suono Cantato}.\newline La testa di nota tradizionale indica la posizione di diteggiatura; la testa di nota tonda, attraversata dal gambo, indica il suono da intonare, in un determinato rapporto intervallare con la posizione di diteggiatura, ma senza relazioni di ottava. \\

\includegraphics[]{../lilypondstuff/LEGENDA/06Cvoceunis.pdf}
& {\scaps Suono Cantato, Unisono}.\newline Quando non è indicata una posizione diversa per lo strumento, voce e strumento sono all'unisono. \\

\end{tabular}}
\end{center}
%\label{default}
\end{table}%

\clearpage
%+++++++++++++++++++++++++++++++++++++

\begin{table}[ht]
%\caption{default}
\begin{center}{\small
\begin{tabular}{m{6,1cm}m{9cm}}

\includegraphics[]{../lilypondstuff/LEGENDA/07Ccluster.pdf}
& {\scaps Multifonico}. \newline È indicata la diteggiatura, non il contenuto armonico. \\

\includegraphics[]{../lilypondstuff/LEGENDA/09Dglissvoce.pdf}
& {\scaps Glissato di voce}. \newline Su una posizione data, la voce attacca con un suono non intonato ed arriva all'unisono con lo strumento mediante un lento glissato. \\

\includegraphics[]{../lilypondstuff/LEGENDA/10Dsoffio.pdf}
& {\scaps Soffio Duro}. \newline Si può produrre un soffio duro stirando la bocca come in un sorriso e facendo defluire l'aria sia nel tubo che attraverso i denti. Alcune posizioni sono indicate sul pentagramma, altre offrono la libertà di scegliere la combinazione di tasti, purché si verifichi una variazione timbrica della porzione di soffio che attraversa lo strumento. \\

\includegraphics[]{../lilypondstuff/LEGENDA/11Dfraseggio.pdf}
& {\scaps Fraseggio Libero}. \newline Fraseggio libero, sfruttando il più possibile l'estensione dello strumento. \\

\includegraphics[]{../lilypondstuff/LEGENDA/12Dreverse.pdf}
& {\scaps Reverse 10''}. \newline Eseguire l'intero pannello in un percorso retrogrado di 10 secondi, cercando di accennare tutte più articolazioni possibili. \\

\end{tabular}}
\end{center}
%\label{default}
\end{table}%

\clearpage

~

\includepdf[pages=-]{../lilypondstuff/score.pdf}

\clearpage

	\fontsize{81}{10}\selectfont{
	\vspace{6cm}454x
	}
	
	\fontsize{19}{10}\selectfont{
	\textit{
		A. quartet of SAX. \& Electronics
	}}

\fontsize{10}{12}\selectfont{}

\section*{Quartetto}

\begin{multicols}{2}
Fin dal principio ho pensato \emph{A.SAX.} come un brano che potesse vivere di un suo tempo specifico. L'idea di analizzare brani del repertorio per sax solo, per poi generare moduli con uno stile specifico, potrebbe portare ad un numero di moduli superiore agli otto attuali e, di conseguenza, aumentarne le combinazioni possibili. 

Ma questo tempo specifico può essere vissuto orizzontalmente o, come spesso accade in musica, verticalmente. L'idea del quartetto nasce da una discussione in merito a questa peculiarità di \emph{A.SAX.} con Pasquale Citera, con un riferimento particolare alla funzione spaziale che un'esecuzione con quattro sax potrebbe avere:

\setlength{\tabcolsep}{12pt}
\renewcommand{\arraystretch}{3}

\begin{center}
\begin{tabular}{c c c}

sax 1 & & sax 2 \\
& S.T.ONE & \\
sax 4 & & sax 3 \\

\end{tabular}
\end{center}
\bigskip

i quattro sassofonisti si dispongono intorno al diffusore omnidirezionale formando una \emph{X} con il diffusore al centro, rivolti verso l'esterno e dando le spalle al diffusore. Il publico, libero di muoversi, circonda l'ensemble di sassofoni. 

Rimangono valide alcune caratteristiche del brano. Il quartetto sceglie gli strumenti per eseguire i moduli o parte di essi. Ogni componente suona, a turno, il modulo \emph{D. SCHEMA PRIMO}, mentre i restanti tre moduli a testa vengono scelti dal quartetto, all'interno dell'\emph{Introduzione} e del \emph{Finale} dati, in cui l'ensemble suona insieme. 

\bigskip
\setlength{\tabcolsep}{7pt}
\renewcommand{\arraystretch}{1.2}

\begin{center}
\begin{tabular}{r c c c c c c}

& \multicolumn{6}{c}{Moduli} \\
\hline
\hline 
sax 1 & Intro & D & & & & Finale \\
\hline
\hline
sax 2 & Intro & & D & & & Finale \\
\hline
\hline
sax 3 & Intro & & & D & & Finale \\
\hline
\hline
sax 4 & Intro & & & & D & Finale \\
\hline
\hline
\end{tabular}
\end{center}

\bigskip


Ogni componente del quartetto non può suonare moduli ripetuti, mentre la ripetizione del modulo interna al quartetto è inevitabile. È possibile scegliere lo stesso modulo per più esecutori contemporaneamente tranne, ovviamente, per il modulo \emph{G} che scorre un esecutore alla volta. La struttura del modulo \emph{G} fa si che al termine ci si trovi a ripercorrere al ritroso l'intero modulo. Seguendo questo filo si collegano i quattro strumentisti nella coincidenza, rispettivamente, di fine e inizio modulo.




\end{multicols}

%\begin{figure}[h!]
%\begin{center}
%\includegraphics[width=0.45\textwidth]{images/IMG_1335.png}
%\caption{Posizioni}
%\label{default}
%\end{center}
%\end{figure}

\end{document}  