%!TEX TS-program = xelatex
%!TEX encoding = UTF-8 Unicode
% !TEX root = ./asax_score_xelatex_UTF8.tex

\section*{Legenda}

%+++++++++++++++++++++++++++++++++++++

\begin{table}[ht]
%\caption{default}
\begin{center}{\small
\begin{tabular}{m{6,1cm}m{9cm}}

\includegraphics[]{../lilypondstuff/LEGENDA/01intro.pdf}
&
{\scaps Suono di denti}.\newline  Appoggiando i denti sulla superficie dell'ancia si generano suoni molti acuti.
Trovare il suono più acuto possibile e tenerlo \emph{fortissimo}, senza vibrare né
modulare l'ampiezza. \\

\includegraphics[]{../lilypondstuff/LEGENDA/02finale.pdf}
&
{\scaps Trillo Grave}.\newline Effettuare il \emph{trillo} più grave possibile, \emph{diminuendo}, fino a
far scomparire il timbro dello strumento e lasciare il suono delle meccaniche.
Il \emph{trillo grave} del saxofono è l'ultimo suono del brano, tutta l'elettronica
deve scomparire prima che il trillo sia consumi. \\

\includegraphics[]{../lilypondstuff/LEGENDA/03Aglissato.pdf}
&
{\scaps Glissato irregolare}.\newline Si procede nella direzione del glissato
alternando intervalli corti a intervalli lunghi. \\

\includegraphics[]{../lilypondstuff/LEGENDA/04Cglissarmonico.pdf}
&
{\scaps Armonico portato}.\newline Tenendo fissa la posizione, \emph{portare} in maniera graduale il suono \emph{armonico} al \emph{fondamentale} e viceversa. \\

\includegraphics[]{../lilypondstuff/LEGENDA/05Cvoce.pdf}
&
{\scaps Suono Cantato}.\newline La testa di nota tradizionale indica la posizione di diteggiatura; la testa di nota tonda, attraversata dal gambo, indica il suono da intonare, in un determinato rapporto intervallare con la posizione di diteggiatura, ma senza relazioni di ottava. \\

\includegraphics[]{../lilypondstuff/LEGENDA/06Cvoceunis.pdf}
& {\scaps Suono Cantato, Unisono}.\newline Quando non è indicata una posizione diversa per lo strumento, voce e strumento sono all'unisono. \\

\end{tabular}}
\end{center}
%\label{default}
\end{table}%

\clearpage
%+++++++++++++++++++++++++++++++++++++

\begin{table}[ht]
%\caption{default}
\begin{center}{\small
\begin{tabular}{m{6,1cm}m{9cm}}

\includegraphics[]{../lilypondstuff/LEGENDA/07Ccluster.pdf}
& {\scaps Multifonico}. \newline È indicata la diteggiatura, non il contenuto armonico. \\

\includegraphics[]{../lilypondstuff/LEGENDA/09Dglissvoce.pdf}
& {\scaps Glissato di voce}. \newline Su una posizione data, la voce attacca con un suono non intonato ed arriva all'unisono con lo strumento mediante un lento glissato. \\

\includegraphics[]{../lilypondstuff/LEGENDA/10Dsoffio.pdf}
& {\scaps Soffio Duro}. \newline Si può produrre un soffio duro stirando la bocca come in un sorriso e facendo defluire l'aria sia nel tubo che attraverso i denti. Alcune posizioni sono indicate sul pentagramma, altre offrono la libertà di scegliere la combinazione di tasti, purché si verifichi una variazione timbrica della porzione di soffio che attraversa lo strumento. \\

\includegraphics[]{../lilypondstuff/LEGENDA/11Dfraseggio.pdf}
& {\scaps Fraseggio Libero}. \newline Fraseggio libero, sfruttando il più possibile l'estensione dello strumento. \\

\includegraphics[]{../lilypondstuff/LEGENDA/12Dreverse.pdf}
& {\scaps Reverse 10''}. \newline Eseguire l'intero pannello in un percorso retrogrado di 10 secondi, cercando di accennare tutte più articolazioni possibili. \\

\end{tabular}}
\end{center}
%\label{default}
\end{table}%